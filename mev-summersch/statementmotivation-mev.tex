%%%%%%%%%%%%%%%%%%%%%%%%%%%%%%%%%%%%%%%%%
% Medium Length Professional CV
% LaTeX Template
% Version 2.0 (8/5/13)
%
% This template has been downloaded from:
% http://www.LaTeXTemplates.com
%
% Original author:
% Trey Hunner (http://www.treyhunner.com/)
%
% Important note:
% This template requires the resume.cls file to be in the same directory as the
% .tex file. The resume.cls file provides the resume style used for structuring the
% document.
%
%%%%%%%%%%%%%%%%%%%%%%%%%%%%%%%%%%%%%%%%%

%----------------------------------------------------------------------------------------
%	PACKAGES AND OTHER DOCUMENT CONFIGURATIONS
%----------------------------------------------------------------------------------------

\documentclass{resume2} % Use the custom resume.cls style

\usepackage[left=0.75in,top=0.6in,right=0.75in,bottom=0.6in]{geometry} % Document margins
\usepackage{hyperref}
\newcommand{\Cyclus}{\textsc{Cyclus}\xspace}%
\usepackage{setspace}



\name{Gwendolyn Chee} % Your name\
\address{gchee2@illinois.edu \\ (217)$\cdot$~904$\cdot$~9057 \\ https://github.com/gwenchee}
\address{\textbf{MeV Summer School Motivation Letter}}

\begin{document}
%----------------------------------------------------------------------------------------
%	EDUCATION SECTION
%----------------------------------------------------------------------------------------
\vspace{1cm}
\onehalfspacing
Introduce myself 

I am a first year Graduate Student in the Nuclear, Plasma and Radiological Engineering department at the University of Illinois Urbana-Champaign. I am a member of Dr. Kathryn Huff's Advanced Reactors and Fuel Cycles (ARFC) research group. The group's research revolves around the usage and development of computational tools to study different types of advanced reactors and their encompassing fuel cycles. 

My research and how i will benefit from MeV summer school 

My research focuses on the fuel cycle and geologic repository modeling. One of the pressing issues currently facing the US nuclear industry is the absence of a nuclear waste repository. Depending on the direction taken by the industry in terms of reprocessing and the deployment of advanced reactors, the composition of the nuclear waste will change. Therefore, by being able to computationally model the valuable types of reactors and their respective waste product compositions, various metrics related to the fuel cycle and repository can be evaluated. 

MeV Summer School will provide me with a better grasp of the theoretical concepts of the various advanced reactors that are currently being developed. It will also give guidance on model development and validation. With this knowledge, I can better develop computational tools to help study the long term effects of each advanced reactor type. I can develop better and more sophisticated tools to evaluate fuel cycles that include different types of advanced reactors and so we have a better quantitative understanding of how they affect environmental, economic and other metrics.

My future career goals is to join a national lab or corporation and continue meaningful work in advocating for nuclear energy and improving the current fuel cycle. Be it, developing 4th generation nuclear reactors that produce less radioactive waste or researching the methods to improve the current nuclear fuel cycle and waste management techniques. Therefore, MeV summer school will better inform and prepare me to meet these career goals. 

\end{document}
