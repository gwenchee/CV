%%%%%%%%%%%%%%%%%%%%%%%%%%%%%%%%%%%%%%%%%
% Medium Length Professional CV
% LaTeX Template
% Version 2.0 (8/5/13)
%
% This template has been downloaded from:
% http://www.LaTeXTemplates.com
%
% Original author:
% Trey Hunner (http://www.treyhunner.com/)
%
% Important note:
% This template requires the resume.cls file to be in the same directory as the
% .tex file. The resume.cls file provides the resume style used for structuring the
% document.
%
%%%%%%%%%%%%%%%%%%%%%%%%%%%%%%%%%%%%%%%%%

%----------------------------------------------------------------------------------------
%	PACKAGES AND OTHER DOCUMENT CONFIGURATIONS
%----------------------------------------------------------------------------------------

\documentclass{resume2} % Use the custom resume.cls style

\usepackage[left=0.75in,top=0.6in,right=0.75in,bottom=0.6in]{geometry} % Document margins
\usepackage{hyperref}
\usepackage{natbib}
\usepackage{bibentry}
\usepackage{enumitem}
\setenumerate[1]{label={[\arabic*]}}
\bibliographystyle{plain}

\name{Gwendolyn J. Chee} % Your name
\address{gchee2@illinois.edu \\ https://github.com/gwenchee } % Your phone number and email

\begin{document}
%----------------------------------------------------------------------------------------
%	EDUCATION SECTION
%----------------------------------------------------------------------------------------
\centering
\iffalse
I am an engineer that is passionate about \\ 
nuclear technology and developing innovative \\
systems to face today's energy challenges.   

Gwen is passionate about nuclear power and developing \\
innovative systems to improve nuclear energy and fuel cycle\\
technologies. She wants to contribute to pressing global\\
energy challenges and environmental sustainability. 
\fi

\raggedright

\begin{rSection}{Education}
	
\begin{tabbing}
PhD \hspace*{1.5 em}\= \textbf{University of Illinois at Urbana-Champaign} \hspace*{5em} \= \hspace*{7.7em} \= 2019 - Present \\
\> Nuclear, Plasma and Radiological Engineering \\
\> Graduate Concentration: Computational Science and Engineering \\
\> \textit{Research focus: Predictive Analytics to Optimize Nuclear Reactor Designs} \\
%	
MS \hspace*{2 em}\= \textbf{University of Illinois at Urbana-Champaign} \hspace*{5em} \= \hspace*{9em} \= 2017 - 2019 \\
\> Nuclear, Plasma and Radiological Engineering \\
\> \textit{Thesis: Sensitivity Analysis of Nuclear Fuel Cycle Transitions}\\
%
BASc \hspace*{2 em}\> \textbf{Queen's University at Kingston, Canada} \> \hspace*{9em} \= 2013 - 2017 \\
\> Engineering Physics, Material Science focus \\
\> \textit{Thesis: Designing a System to Gaseous Hydrogen Charge Zirconium Alloys }
\end{tabbing}

\end{rSection}

%----------------------------------------------------------------------------------------
%	WORK EXPERIENCE SECTION
%----------------------------------------------------------------------------------------

\begin{rSection}{Research Experience}

\begin{rSubsection}{University of Illinois at Urbana-Champaign}{2017 - Present}{Research Assistant, Advanced Reactors and Fuel Cycles}{Urbana, IL}
Advisor: Professor Kathryn D. Huff \\
Developed demand-driven deployment algorithms for \textsc{Cyclus}, and coupled \textsc{Cyclus} with Dakota 
to perform sensitivity analysis on nuclear fuel cycle transitions. Currently working on leveraging generative 
machine learning algorithms to optimize modular High Temperature Gas Cooled Reactor designs. 
\end{rSubsection}

\begin{rSubsection}{Argonne National Laboratory}{May 2019 - Aug 2019}{Research Aide}{Lemont, IL}
Advisor: Dr. Bo Feng \\ 
Coupled Dymond with Dakota to perform sensitivity analysis on nuclear fuel cycle transitions.
\end{rSubsection}

\begin{rSubsection}{Queen's University at Kingston}{2016 - 2017}{Research Assistant, Nuclear Materials Research Group}{Kingston,ON}
Advisor: Professor Mark Daymond \\
Designed a Sieverts Apparatus to gaseously charge hydrogen gas into zirconium alloys to 
mimic hydrogen embrittlement of zirconium alloys used in nuclear reactors.
\iffalse
The design is being implemented at Reactor Materials Testing Laboratory to test how hydrogen embrittled zirconium alloys respond in nuclear reactor conditions
 Application to nuclear industry: zirconium alloys used in nuclear reactors succumb to hydrogen embrittlement during its lifetime, therefore, it is important to be able to replicate the conditions in nuclear reactors, so as to study its end-of-life conditions
\fi
\end{rSubsection}

\begin{rSubsection}{National University of Singapore}{May 2016 - Aug 2016}{Research Assistant, Centre for Advanced 2D Materials}{Singapore}
Advisor: Professor Jens Martin \\	 
Developed a MATLAB script to study the effect of Berry Curvature on electrons in graphene and the 
effects of changing the geometry of graphene devices on their electric fields. 
\iffalse
Both programs were used to assist graduate students in their design of nano-graphene devices
\fi
\end{rSubsection}

\begin{rSubsection}{Nanyang Technological University}{May 2015 - Aug 2015}{Research Assistant, Polymeric Biomaterials Group}{Singapore}
Conducted experiments to characterize nanoparticle enhanced polymer materials to determine the material 
combination that best increases the mechanical properties of biodegradable heart stents.  
\end{rSubsection}

\end{rSection}

\begin{rSection}{Professional Service}

	\begin{rSubsection}{U.S. Women in Nuclear}{2018 - Present}{President}{Urbana, IL}
	Leads the UIUC WiN chapter to uplift the mission of professional development, 
	educational outreach, and a sense of community amongst our members 
	(\href{https://github.com/gwenchee/wincv}{WiN CV}).
	\end{rSubsection}
	
	\end{rSection}

%----------------------------------------------------------------------------------------
%	ENGINEERING EXPERIENCE
%----------------------------------------------------------------------------------------

\begin{rSection}{Engineering Experience}
	
	\begin{rSubsection}{Engineering Physics Capstone Project}{2016}{Self Sorting Recycling Bin}{Kingston,ON}
		 Developed a neural network to sort between recycling and garbage through image recognition and sound profiling.
		 Led the mechanical team to prototype the physical design which used feedback from the neural network to 
		 physically separate the items. 
	\end{rSubsection}
\iffalse
	\begin{rSubsection}{Engineering Physics Design Project}{2015}{Photodiode Research Sensor}{}
		 Designed and fabricated a research sensor used to assess the properties of photodiodes. Information gathered included efficiency, current and voltage under illuminated and darkened conditions using an Arduino, Matlab and LabVIEW
	\end{rSubsection}

	\begin{rSubsection}{Engineering Design and Practice II}{2014}{Nuclear Waste Gamma Radiation Detector}{Kingston,ON}
		 Prototyped a Nuclear Waste Gamma Radiation Detector. Through the process of material selection and decision making, the final product was designed for use in areas surrounding the Deep Geologic Repository in Ontario
	\end{rSubsection}

	\begin{rSubsection}{Wirecard AG}{Summer 2014}{Summer Technology Intern}{Singapore}
	 Redesigned Wirecard's payment processing webpage using HTML and CSS
	 It is currently used for redirecting online payments to Wirecard's payment processing service	
	\end{rSubsection}
\fi
\end{rSection}

%----------------------------------------------------------------------------------------
%	TEACHING EXPERIENCE
%----------------------------------------------------------------------------------------

\begin{rSection}{Teaching Experience}
	
	\begin{rSubsection}{Queen's University at Kingston}{2015 - 2017}{Teaching Assistant, Physics Department}{Kingston,ON}
		 Conducted weekly help sessions for students who required extra guidance in first year physics courses (PHYS 104/106).
	\end{rSubsection}

\end{rSection}

\nobibliography{cv}
\begin{rSection}{Conference Proceedings}
	\begin{enumerate}[series=myexample]
		\item \bibentry{chee_demonstration_2019}
		\item \bibentry{chee_simulation_2019}
		\item \bibentry{chee_validation_2018}
		\item \bibentry{chee_numerical_2018}
	\end{enumerate}
\end{rSection}

\begin{rSection}{Technical Reports}
	\begin{enumerate}[resume=myexample]
		\item \bibentry{chee_transition_2019}
	\end{enumerate}
\end{rSection}
%----------------------------------------------------------------------------------------
%	TECHNICAL STRENGTHS SECTION
%----------------------------------------------------------------------------------------

\begin{rSection}{Selected Awards and Recognition}
Women in Nuclear Chapter Excellence Award \hspace{53.5ex} 2019
Queens University Deans Scholar \hspace{61ex} 2014-2017
\end{rSection}

\begin{rSection}{Scientific Computing Skills}

\begin{tabular}{ @{} >{\bfseries}l @{\hspace{6ex}} l }
Languages & bash, Python, C++, XML, HTML\\
Build Systems & make, cmake\\  
Databases & SQL \\
Test Frameworks & nose, pytest\\
Other Tools &  \LaTeX, Mathematica, Jupyter, MatLab, Dakota\\
Nuclear Software & \textsc{Cyclus} , PyNE \\
\end{tabular}

\end{rSection}

\end{document}
