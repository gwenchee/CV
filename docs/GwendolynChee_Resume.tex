\documentclass[11pt,letterpaper]{article}
\usepackage[utf8]{inputenc}
\usepackage[letterpaper,margin=1in]{geometry}
\usepackage{verbatim}
\usepackage{multicol}

\newlength\tindent
\setlength{\tindent}{\parindent}
\setlength{\parindent}{0pt}
\renewcommand{\indent}{\hspace*{\tindent}}

\begin{document}
{\Huge\textbf{Jordan Leung}} \hfill  jordan.leung@mail.utoronto.ca
\vspace{0.1cm}
\hrule
\vspace{0.25cm}
\noindent 3 Moulton Court, Courtice ON, L1E 2W4 \hfill 289-928-3350
\vspace{0.25cm}


{\large\textbf{EDUCATION}} % EDUCATION HEADING
\vspace{0.1cm}
\hrule
\vspace{0.25cm}
\textbf{University of Toronto}, Toronto ON \hfill (2017 - 2019) \\
\textit{M.A.Sc.} Aerospace Engineering, GPA 4.0/4.0 \\
\textit{Research Area:} Aircraft Flight Simulation and Modelling

\vspace{0.3cm}
\textbf{Queen's University}, Kingston ON \hfill (2013 - 2017) \\
\textit{B.Sc.} Engineering Physics, GPA 4.1/4.3 \\
\textit{Sub-option:} Mechanical Engineering \\


% RESEARCH EXPERIENCE HEADING
{\large\textbf{RESEARCH EXPERIENCE}}
\vspace{0.1cm}
\hrule
\vspace{0.25cm}

\textbf{Graduate Research Assistant}  \hfill  (Sept. 2017 - May 2019) \\
Vehicle Simulation Laboratory, Professor P. R. Grant \\ University of Toronto Institute for Aerospace Studies, Toronto ON  \\
\vspace{-0.6cm}
\begin{itemize}
	\item  Researching methods for developing representative post-stall models of transport aircraft using certification flight test data for use in upset recovery simulation
\end{itemize}
\vspace{-0.2cm}

% THESIS PROJECT
\textbf{Undergraduate Research Assistant} \hfill  (Sept. 2016 - Apr. 2017) \\ Rival Lab, Professor D. E. Rival \\ Queen's University Department of Mechanical \& Materials Engineering, Kingston ON
  \\
\vspace{-0.6cm}
\begin{itemize}
	\item Conducted research in the field of experimental fluid dynamics aiming to simulate the aerodynamic effects of vertical gusts in flight by moving a test model through water
	\vspace{-0.25cm}
	\item Performed Computational Fluid Dynamics (CFD) simulations to compare the flow development of a canonical gust to different dynamically scaled model motions
	\vspace{-0.25cm}
	\item Familiarized myself with unsteady fluid dynamic theory such as Theodorsen's theory, indicial response theory, and unsteady wake development
\end{itemize}
\vspace{-0.2cm}


% RESEARCH ASSISTANT
\textbf{Undergraduate Research Assistant}  \hfill  (May 2016 - Aug. 2016) \\
 Professor D. E. Rival \\ Queen's University Department of Mechanical \& Materials Engineering, Kingston ON
\vspace{-0.2cm}
\begin{itemize}
	\item  Conducted research in the field of experimental fluid dynamics aiming to characterize the aerodynamics of manoeuvring delta wings under the effects of gust
	\vspace{-0.25cm}
	\item Created MATLAB models to simulate the dynamics of a test manoeuvre, and interfaced pressure, force, and motion sensors using LabVIEW
\end{itemize}
\vspace{-0.1cm}



{\large\textbf{PROFESSIONAL AND EXTRACURRICULAR EXPERIENCE}} % PROFESSIONAL EXPERIENCE HEADING
\vspace{0.1cm}
\hrule
\vspace{0.25cm}

% AERO DESIGN

\textbf{Team Captain}, Queen's University SAE Aero Design Team \hfill (2013 - 2017)
\vspace{-0.2cm}
\begin{itemize}
	\item Lead a team of over 40 engineering students to design and construct two RC planes to compete in the annual SAE Aero Design competition amongst universities across Canada, the United States, and several other countries throughout the world
	\vspace{-0.25cm}
	\item Designed and modelled aspects of the aircraft in XFOIL,  XFLR5, MATLAB, and SolidWorks
	\vspace{-0.75cm}
	\item Responsible for overseeing all duties related to technical design, project management, financial management, administration, sponsorship, and recruitment
	\vspace{-0.25cm}
	\item 	 Refined time management skills by devoting a significant amount of time to design team projects during the school year, while also maintaining a high academic standing
	\vspace{-0.25cm}
	\item Previous positions: Stability \& Tail Design Lead, Aerodynamics Team Member
\end{itemize}


\textbf{UP Express Summer Student} \hfill (May 2015 - Sept. 2015) \\
Metrolinx, Toronto ON
\vspace{-0.2cm}
\begin{itemize}
	\item Interacted with guests on the UP Express platform, responding to inquiries, improving guest satisfaction and acting as the face and voice of the company to the public
	\vspace{-0.25cm}
	\item Performed an analysis of the ridership frequency and diversity of UP Express, and developed a methodology to manually record ridership data
\end{itemize}





{\large\textbf{TEACHING EXPERIENCE}}
\vspace{0.1cm}
\hrule
\vspace{0.3cm}

\textbf{Mechanics and Dynamics} (ENPH 225), Queen's University \hfill (Jan. 2017 - Apr. 2017)
	\vspace{-0.25cm}
\begin{itemize}
	\item Marked midterms and assignments for the 2nd-year Mechanics and Dynamics course
\end{itemize}
	\vspace{-0.25cm}
\textbf{Numerical Methods} (APSC 100), Queen's University \hfill (Sept. 2016 - Dec. 2016)
	\vspace{-0.25cm}
\begin{itemize}
	\item Co-instructed tutorials teaching numerical analysis using MATLAB to 1st-year students
	\vspace{-0.25cm}
	\item Marked assignments and quizzes on engineering design and numerical methods
\end{itemize}








\begin{comment}
% PAYLOAD DESGIN 4th year
\textbf{Roll Control of Sounding Rocket Payload} \hfill (Sept. 2016 - Dec. 2016) \\ \textit{4$^{th}$ Year Engineering Design Project}
\vspace{-0.15cm}
\begin{itemize}
\item Worked in a group to design a CubeSat sized payload to be ejected out of a sounding rocket at an apogee of 10,000 ft that descends and controls its roll rate using  servo-actuated fins and a PID control method
\vspace{-0.25cm}
\item Components of the payload include an altitude triggered parachute deployment mechanism, radio-frequency telemetry and two-way communications, GPS tracking,  a drag measurement system, and PID controlled fins, all of which are controlled through an Arduino
\vspace{-0.25cm}
\item Personally responsible for implementing the PID control, sensor integration, and two-way radio-frequency communications and telemetry of the payload
\end{itemize}


% PHOTODIODE 3rd year
\textbf{Photodiode Performance Evaluator} \hfill (Jan. 2016 - Apr. 2016)
\\ \textit{3$^{rd}$ Year Engineering Design Project}
\vspace{-0.15cm}
\begin{itemize}
\item Designed hardware and software capable of extracting  performance parameters (e.g. efficiency) of a photodiode using circuit components such as operational amplifiers and low-pass filters and an Arduino microcontroller interfaced with LabVIEW
\end{itemize}
\end{comment}









{\large\textbf{SELECTED AWARDS AND RECOGNITIONS}} % ENG PROJECTS HEADING
\vspace{0.1cm}
\hrule
\vspace{-0.2cm}
\begin{itemize}
	\item James H. Rattray Memorial Scholarships in Applied Science, Queen's University \hfill (2016)
	\vspace{-0.25cm}
	\item Natural Science and Engineering Research Council of Canada Undergraduate Student\\ Research Award, Queen's University \& NSERC \hfill (2016)
	\vspace{-0.25cm}
	\item Osler, Hoskin \& Harcourt LLP Achievement Award \hfill (2013 \& 2014 \& 2015 \& 2016)
	\vspace{-0.25cm}
	\item Queen’s University Dean’s Scholar \hfill (2014 \& 2015 \& 2016 \& 2017)
	\vspace{-0.25cm}
	\item Physics Department Award, Queen's University \hfill (2015)
\end{itemize}


\vspace{0.2cm}
{\large\textbf{TECHNICAL SKILLS}}
\vspace{0.1cm}
\hrule
\vspace{-0.2cm}
\begin{multicols}{2}
	\begin{itemize}
		\item SolidWorks CAD Modelling
		\item SolidEdge CAD Modelling
		\item MATLAB Programming and Modelling
		\item Arduino \& C Programming
		\item LabVIEW Programming
		\item XFOIL \& XFLR5
		\item Aircraft design, performance, and dynamics analysis
		\item Aircraft system identification and modelling
		\item \LaTeX

	\end{itemize}
\end{multicols}



\end{document}
